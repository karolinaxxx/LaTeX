\documentclass[a4paper,12pt]{article}
\usepackage{amsmath}
\usepackage{amssymb}
\usepackage{algorithmic}
\usepackage[T1]{fontenc}
\usepackage[polish]{babel}
\usepackage[utf8]{inputenc}
\usepackage{lmodern}




\selectlanguage{polish}

%opening
\title{TeX}

\author{}

\begin{document}
\maketitle
\begin{abstract}

\end{abstract}

\section{Formuły matematyczne w TeXu}
Przetrenuj używanie w TeXu matematycznych formuł i symboli z rodziału 1 po czym 
wykonaj polecenie z rozdziału 2.                
\subsection{Zapis Matematyczny}
\subsubsection{Tryb matematyczny}
Tryb matematyczny 'inline'-wzory pisane w lini tekstu wstawiamy przy pomocy symbolu dolara \$
 (wzór wpisujemy w pojedyńcze dolary).\newline 
Ułamek w tekście $\frac{1}{x}$
Oto równanie $c^{2}=a^{2}+b^{2}$
\newline
Tryb matematyczny z zastosowaniem podwujnych dolarów.\newline
Ułamek 
$$\frac{1}{x}\\$$
Oto równanie 
$$c^{2}=a^{2}+b^{2}$$

Tryb matematyczny z użyciem struktury 'equation'\newline
Ułamek
\begin{equation}
\frac{1}{x}
\label{eq:rónanie1}
\end{equation}
Oto równanie
\begin{equation}
c^{2}=a^{2}+b^{2}
\label{eq:równanie2}
\end{equation}

Można odnieść się do powyższych wzorów wykorzystując polecenie 'qref{etykieta}'.
Ułamek ma numer (1) a równanie ma numer (2)
Wiele wzorów w ramach jednego środowiska matematycznego, przy pomocy znaku `and'
możemy dokonać wyrównania równań:

\begin{align}
\label{eq:partialLW}
\frac{\partial \mathcal L (w,b,\xi,\alpha,\beta)}{\partial w}=0 & \Rightarrow w 
-\sum_{i=1}^n\alpha_i y_i x_i=0,\\
\label{eq:partialLXi}
\frac{\partial \mathcal L (w,b,\xi,\alpha,\beta)}{\partial \xi_i}=0 & \Rightarrow
C-\alpha_i-\beta_i=0,\\
\label{eq:partialLB}
\frac{\partial \mathcal L (w,b,\xi,\alpha,\beta)}{\partial b}=0 & \Rightarrow 
\sum_{i=1}^n\alpha_i y_i=0.
\end{align}

Zad.1.
Przestudiuj trzy powyższe przypadki,zwróć uwagę na róznice w wyświetlaniu i możliwości pózniejszego odwołania się do równania. Przepisz je do latex'a i spróbuj odbiesc  się do równania zdefiniowanych przy pomocy 'equation'.
\subsubsection{Indeks górny i dolny} 

Do tworzenia indeksu górnego uzywamy operatorów \^ \ oraz podkreślenia \_ \newline
Indeks górny 
$$x^{y} \ c^{x}\  2^{e} \ A^{2 \times 2}$$\\
Indeks dolny
$$x_y \ a_{ij}\ x_{i} $$
Oba indeksy
$$x^{2}_{i}\ x^{kj}_{i2}\ a^{k}_{ij}$$

\subsubsection{Podstawowe funkcjie}
Pierwiastek, ułamek
$$\sqrt{\frac{2^{n}}{2^{n}}} \neq \sqrt [\frac{1}{3}]{1+n} $$
Zad.2.
Przepisz powyzsze przykłady zwróć uwagę na odstępy pomiędzy wyrazeniami.Napisz formuły tworzące poniższe przykłady:
$$\frac {2^{k}}{2^{k+2}} $$
$$\frac{x^2}{2^{(x+2)(x-2)^{3}}}$$\newline
$$\vec{x}=[x_1,x_2,\dots x_N]$$\newline
$$\log_{2}2^{8}=8$$\newline
$$\sqrt[^3]{e^x-\log_{2}x}$$

\subsubsection{Duże operatory matematyczne}
$$\sum \ \sum_{i=1}^{10}x_{i} \ \prod \ \coprod \ \int \ \oint \ \bigcap \ \bigcup
\ \bigsqcup \ \bigvee \ \bigwedge \ \bigodot \ \bigotimes \ \bigoplus \ \biguplus$$
\newline
Operatory wielokrotne, np.podwójne całki.\newline
$$\int\int_D\,\mathrm{d} x \,\mathrm{d} y$$\newline
$$\int\!\!\int_D\,\mathrm{d}x \,\mathrm{d}y$$

Zad 3.\newline\newline
\begin{equation}
\int^{+\infty}_{-\infty}\,\mathrm{e}^{-x^{2}}\,\mathrm{d}x 
\end{equation}
$$\sum^N_{k=1}\frac{k\ast\sin(k)}{2^k}$$
$$\sum^N_{i=1}\sum^N_{j=1}\ i\ast j$$


\subsubsection{Dwumiany}
choose- dostawia nawiasy\newline atop-nie dostawia nawiasy.
\begin{equation}
{n\choose k}\qquad {x \atop y+2}
\end{equation}
\subsection{Nawiasy}
Lewy i prawy automatycznie się dostosowujący.
\begin{equation}
1+\left(\frac{1}{1-x^2} 
\right)^3
\end{equation}
Sterowanie nawiasami samemu.\newline
$\Big( (x+1)(x-1) \Big)^{2}$\\
$\big(\Big(\bigg(\Bigg($\qquad
$\big\}\Big\}\bigg\}\Bigg\}$\qquad
$\big\|\Big\|\bigg\|\Bigg\|$

\subsubsection{Akcenty}
$$\hat{a} \ \check{b}\ \breve{c}\ \acute{d} \ \grave{e}\ \tilde{f} \ \bar{g} \ \vec{h} \ \dot{m}
\ \ddot{n}$$
$$\widetilde{aaa} \ \widehat{bbb}\ \overleftarrow{ccc} \ \overrightarrow{ddd}\ \overline{eee} \ \overbrace{fff} \ \underbrace{ggg} \ \underline{hhh} \ \sqrt{iii}  \ 
\sqrt[n]{jjj} \ \frac{kkkk}{}$$

\subsection{Alfabet Grecki}
$$\Gamma \ \Delta \ \Theta \ \Xi \ \Pi \ \Sigma \ \Upsilon \ \Phi \ \Psi \ \Omega$$
$$\alpha \ \beta \ \gamma \ \delta \ \epsilon \ \varepsilon \ \zeta \ \eta \ \theta \
\vartheta \ \iota \ \kappa \ \lambda \ \mu \ \nu \ \xi \ \ o \pi \varpi \ \rho \ \varrho \ \sigma 
\ \varsigma  \ \tau \ \upsilon \ \phi \ \varphi \ \chi \ \psi \ \omega  $$

\subsection{Symbole}
$$\aleph \ \hbar \ \imath \ \jmath \ \ell \ \wp \ \Re \ \Im \ \prime \ \emptyset \ \angle \ \infty \ \partial \ \nabla \ \triangle \ \forall \ \exists \ \neg \ \surd  \ \top \ \bot \ \backslash$$
$$\flat \ \natural \ \sharp \ \| \ \clubsuit \ \diamondsuit \ \heartsuit \ \spadesuit \ \dag \ \ddag\ \S \ \P \ \copyright \ \pounds \ \checkmark \ \maltese \ \circledR \ \yen \ \ulcorner \ \urcorner \ \llcorner \ \lrcorner \ \diamond \ \mho \ \Box \ \cdot \ \ldots \ \cdots \ \vdots \ \ddots$$

\subsection{Nawiasy}
$$( \ [ \{ \ \lfloor \ \lceil \ \langle \  /\  |\  )\  ]\  \} \  \rfloor \rceil \ \rangle \ \backslash \ \| \ \uparrow \ \updownarrow \ \Uparrow \ \Downarrow \ \Updownarrow \ \qquad \ \quad \ \! \ \, \ \: \ \; \ $$ 

\subsection{Znaki}
\subsection{Inne symbole}
\subsection{Użycie struktóry array}
$$e'_{ij}=
\left\{
   \begin{array}{c}
	   e_{ij}\ {\rm gdy}\ d(x_i) \neq d(x_j) \\
		\phi\ {\rm gdy}\ d(x_i)=d(x_j). \\
 \end{array} \right.$$

\begin{equation}
\mathbf{X}=
\left [ \begin{array}{ccc}
x_{11} & x_{12} & \ldots \\
x_{21} & x_{22} & \ldots \\
\vdots & \vdots & \ddots 
\end{array} \right ]
\end{equation}

\begin{equation}
y = \left\{ \begin{array}{ll}
a & \textrm{jeżeli} \ d>c\\
b+x & \textrm{jeżeli} \ d\leq c\\
1 & \textrm{jeżeli} \ d=0
\end{array} \right.
\end{equation}

\subsection{Uzycie środowiska algorithmic}
Potrzebne jest dodanie pakietu "usepackage\{algorithmic\}'
\begin{algorithmic}
\STATE{22) Procedure}
\STATE{Input data}
\STATE{$A' \leftarrow \emptyset$}
\STATE{$iter \leftarrow 0$}
\FOR{i=1,2...,card\{A\}}
\FOR{j=1,2...,k}
\STATE{$S^{c_j}(a)=S_{i}^{c_j}(a)$}
\IF{$a \not \in A'$}
\item{$A' \leftarrow\ a$}
\item{$iter \leftarrow iter+1$}
\IF{$iter=fixed\ number\ of\ the\ best\ genes$}
\item{BREAK}
\ENDIF
\ENDIF
\ENDFOR
\IF{$iter=fixed\ number\ of\ the\ best\ genes$}
\item{BREAK}
\ENDIF
\ENDFOR
\RETURN{$A'$}
\end{algorithmic}
%rozdział 2
\section{Polecenie do wykonania}
%równanie 1 
\begin{equation}
\lim_{n\rightarrow \infty} \sum_{k=1}^n \frac{1}{k^2}=\frac{\pi^2}{6}\\
\end{equation}
\newline
% drugie równanie
\begin{equation}
\prod^{n=i^2}_{i=2}=\frac{\lim^{n \rightarrow 4}(1+\frac{1}{n})^n}{\sum k (\frac{1}{n})}\\
\end{equation}
\newline
%trzecie równanie
\begin{equation}
\int^{\infty}_2 \frac{1}{\log_{2}x}\ \mathrm{d}x = \frac{1}{x} \sin{x}=1-\cos^2(x)\\
\end{equation}
\newline
%array
\begin{equation}
\left[ \begin{array}{cccc}
a_{11} & a_{12} & \ldots & a_{1K}\\
a_{21} & a_{22} & \ldots & a_{2K}\\
\vdots & \vdots & \ddots & \vdots\\
a_{K1} & a_{K2} & \ldots & a_{KK}
\end{array} \right ] \ast
%pierwsza kolumna po gwiastce
\left[ \begin{array}{c}
x_1\\
x_2\\
\vdots\\
x_K
\end{array} \right ]=
% trzecia kolumna
\left [ \begin{array}{c}
b_1\\
b_2\\
\vdots\\
b_K
\end{array} \right ]
\end{equation}
\newline
%równanie 5
\begin{equation}
(a_1=a_1(x)) \wedge (a_2=a_2(x)) \wedge \ldots \wedge(a_k=a_k(x)) \Rightarrow (d=d(u))
\end{equation}
\newline
%6
\begin{equation}
\left[x\right ]_A=\{y \in\mathbf U : a(x)=a(y).\forall a \in \mathbf A\}, {\textrm where\ the\ central\ object}\ x \in \mathbf U
\end{equation}
\newline
%7
\begin{equation}
g(u,r)=\{ v\in \mathbf U :\frac{card\{IND(u,v)\}}{card\{A\}|}\geq r\}
\end{equation}
\newline
%8
\begin{equation}
{\textrm where},IND(u,v)=\{a \in \mathbf A : a(u)=a(v)\}
\end{equation}
\newline
%9
\begin{equation}
{\mathbf T}:[\ 0,1]\times [\ 0,1]\rightarrow [\ 0,1]\ ,
\end{equation}
%10
\newline
\begin{equation}
\cos(2 \theta)=cos^2{\theta}-\sin^2{\theta}
\end{equation}
%11
\newline
\begin{equation}
\lim_{x\rightarrow \infty} {\textrm exp}(-x)=0
\end{equation}
% 12 wzory z silnią 
\newline
\begin{equation}
\frac{n!}{k!(n-k)!}={n \choose k}
\end{equation}
%13AB s
%sprawdzić jak zapisać | tak żeby była równa z nawiasem
\newline
\begin{equation}
P\bigg ( A=2 \mid \frac{A^2}{B} >4 \bigg)
\end{equation}
%14
\begin{equation}
S^{Ci}(a)=\frac{(\overline{C}^u_i - \hat{C}^a_i)^2}{Z_{\bar{C}^{a^2}_i}+Z_{\hat{C}^{a^2}_i}},a \in \mathbf A.
\end{equation}
%15
\begin{equation}
C^a_i=\{a(u) : u \in \mathbf U \quad and \quad d(u)=c_i \}.
\end{equation}
%16
\newline
\begin{equation}
A_{c_i}(a)=C_i^a \wedge_\varepsilon\{ U\backslash C^a_i \}
\end{equation}
%17
\newline
\begin{equation}
w(u_q,v_p)=w(u_q,u_p) + \frac{\mid a(u_q)_a(v_p)\mid}{(max\_attr_a-min\_attr_a)\ast \varepsilon}
\end{equation}
%18
\newline
\begin{equation}
c'_ij= \left \{
\begin{array}{c}
c_{ij} \rm \ gdy \ d(x_i) \neq d(x_j)\\
\phi \rm\ gdy \ d(x_i) = d(x_j).
\end{array} \right.
\end{equation}
%algorytm
\newline
\begin{algorithmic}
\STATE{Procedura}
\STATE{Imput data}
\STATE{$A' \leftarrow \emptyset$}
\STATE{$iter \leftarrow 0$}
\FOR{i=1,2,...,card\{A\}}
\FOR{j=1,2,...,k}
\STATE{$F^{c_j}(a)=F^{c_j}_i(a)$}
\IF{$a \not \in A'$}
\item{$A' \rightarrow a$}
\item{$iter \leftarrow iter +1$}
\IF{$iter=fixed\ number\ of\ the\ best\ genes $}
\item{BREAK}
\ENDIF
\ENDIF
\ENDFOR
\IF{$iter=fixed\ number\ of\ the\ best\ genes $}
\item{BREAK}
\ENDIF
\ENDFOR
\RETURN{$A'$}
\end{algorithmic}
%
$$
S^{c_1}_1 >S^{c_1}_2(a)  > \ldots  > S^{c_1}_{card\{A\}}(a)$$\newline
$$
S^{c_2}_1(a)  >S^{c_2}_2 >\ldots >S^{c_2}_{card\{A\}}(a)$$\newline
 $$\vdots\\$$\newline
$$S^{c_k}_1(a)  >S^{c_k}_2(a)  >\ldots  >S^{c_k}_{card\{A\}}(a)$$
\end{document}