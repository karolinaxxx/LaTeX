\documentclass{beamer}
%%\usepackage[MeX]{polski}
%%\usepackage[cp1250]{inputenc}
\usepackage[]{geometry}
\usepackage{polski}
\usepackage[utf8]{inputenc}
\usepackage{makeidx}
\usepackage[tableposition=top]{caption}
\usepackage{beamerthemesplit}
\usepackage{algorithmic}
\usepackage{graphicx}
\usepackage{array}
\usepackage{enumerate}
\usepackage{multicol}
\usepackage{multirow}
\usepackage{amsmath} %pakiet matematyczny
\usepackage{amssymb} %pakiet dodatkowych symboli
\title{Puchar Karpat w skokach narciarskich}
\author{Karolina Ścibek}
\date{\today}

\begin{document}
\section{Puchar Karpat w skokach narciarskich}

\maketitle
\begin{frame}
\frame
{
\frametitle{Puchar Karpat w skokach narciarskich}
}
\begin{itemize}
\item 
Puchar Karpat w skokach narciarskich (ang. FIS Carpathian Cup (w sezonie 2013/2014) lub FIS Carpath Cup (od sezonu 2014/2015)) – cykl zawodów juniorskich w skokach narciarskich dla państw związanych z Karpatami organizowany pod egidą Międzynarodowej Federacji Narciarskiej. Rozgrywany od sezonu 2013/2014 w kategoriach mężczyzn i kobiet.

\end{itemize}
\end{frame}
\begin{frame}
\frame
{
\frametitle{Pucharu Karpat}
}
\begin{itemize}
\item
Cykl Pucharu Karpat utworzono latem 2013 roku, a przy jego tworzeniu wzorowano się na cyklu Alpen Cup. Impulsem do utworzenia Pucharu Karpat był zorganizowany w 2013 roku Redyk Karpacki, a jego głównym inicjatorem był przewodniczący Komisji Międzynarodowej Federacji Narciarskiej do spraw planowania kalendarza Paul Ganzenhuber[3].
\item 
Jednym z powodów powstania Pucharu Karpat była także odmowa prawa startów w zawodach Alpen Cup skoczków między innymi z Polski (startowali oni w tych zawodach tylko w sezonie 2009/2010).
\end{itemize}
\end{frame}
\begin{frame}
\frame
{
\frametitle{Puchar Karpat}
}
Do udziału w Pucharze Karpat dopuszczono początkowo przedstawicieli 10 krajów: Bułgarii, Czech, Kazachstanu, Polski, Rosji, Rumunii, Słowacji, Turcji, Ukrainy i Węgier. W każdym z konkursów poszczególne kraje mogą wystawić maksymalnie 15 zawodników, a gospodarze zawodów dodatkowo kolejnych 5 (razem 20 osób), których wiek, liczony w sposób rocznikowy, nie może przekraczać 20 lat. Kalendarz  sezonu dzieli się  (periody): letni  i zimowy . Zawody rozgrywane są na skoczniach, których punkt konstrukcyjny mieści się w przedziale od 60 do 100 metrów. Dopuszczalne są trzy rodzaje konkursów: indywidualne (kobiet i mężczyzn), drużynowe (kobiet i mężczyzn) i mieszane.
\end{frame}
\begin{frame}
\frame
{
\frametitle{sezon$(2015 /\ 2016)$}
}
\begin{itemize}
\item
Od trzeciej edycji $(sezon 2015 /\ 2016)$planowane jest rozszerzenie prawa startu w zawodach na przedstawicieli wszystkich krajów, które nie mogą brać udziału w zawodach Alpen Cupu, a także wprowadzenie zapisu, iż wszyscy uczestnicy konkursu, niezależnie od zajętego po 1. serii miejsca, będą mieli prawo oddania dwóch zaliczanych do końcowego wyniku skoków. 
\item
W kolejnych sezonach planowane jest także rozpoczęcie rozgrywania analogicznych cykli zmagań o Puchar Karpat w kombinacji norweskiej i narciarstwie alpejskim.
\end{itemize}
\end{frame}

\section{Zwycięzcy}
\begin{frame}
\frame
{
\frametitle{Zwycięzcy}
}
\begin{itemize}
\item
Sezon I- rozgrywania cyklu Pucharu Karpat w skokach narciarskich w klasyfikacji generalnej wśród mężczyzn zwyciężył Stefan Blega przed Mateuszem Kojzarem i Szymonem Szostokiem , a wśród kobiet wygrała Carina Militaru przed Dianą Trambitas i Magdaleną Pałasz
\item
Sezon II
\item
SezonIII
\item
SezonIV
\end{itemize}
\end{frame}


\frame
{
\begin{table}
\caption{Mężczyźni}
\begin{tabular}{c|c|c|c}
Sezon&1.miejsce&2.miejce&3.miejsce\\
\hline
$2013/\ 14$ &Stefan Blega & Mateusz Kojzar& \\
\hline
$2014/\ 15$& Stefan Blega& Valentin Tatu&Robert Buzescu\\
\hline
\multirow{2}{*}{$2015/\ 16$ } &Muhammet Irfan & Fatih Arda Ipcioglu&Munir Gungen\\
$$&Cintimar& $$& $$\\
\hline
$2016/\ 17$ & Stefan Blega& Sorin Mitrofan& Mihnea Spulber\\

\end{tabular}
\end{table} 
}

\begin{thebibliography}{3}
\bibitem{Ogrocka} Sonia Ogrocka: Szczyrk zaprasza na pierwsze zawody Pucharu Karpat w skokach (pol.). nicesport.pl, 2013-09-18. [dostęp 2014-02-20].
\bibitem{} Warto promować Karpaty. „Głos Ludu”. 106 (LXVIII), s. 3, 2013-09-07 (pol.).
\bibitem{} Katarzyna Służewska: Apoloniusz Tajner dla SP: “Ciągle pojawiają się młodzi, trzeba się uczyć nowych nazwisk” (pol.). skokipolska.pl, 2014-09-22. [dostęp 2014-09-22].
\bibitem{Kosman} Alicja Kosman: Polacy najlepsi w klasyfikacji generalnej FIS Carpath Cup (pol.). pzn.pl, 2014-09-26. [dostęp 2014-09-26].

\end{thebibliography}









\end{document}